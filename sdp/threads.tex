\chapter{Threads}
Processes do not share memory and they can only communicate through a file, i.e.\@ a pipe. The child can communicate exiting status to the parent (just a single byte). The creation of a child process is a costly operation in terms of space, which has to be duplicated, and time, because it is a slow operation. The management of multiple processes requires scheduling and an expensive context switching.

A thread is a \textit{lightweight process}. The thread model allows a program to control multiple different flows of operations that overlap in time. Threads creation and management require a reduced costs with respect a process implementation, they share a single address space (i.e.\@ variables) and the context switch among threads is faster.

A thread is statically represented by a function, it has its own stack and therefore its own variables declared inside it.

\section{POSIX threads}
POSIX threads or Pthreads is the standard UNIX library for threads, defined for C language. Using this library, a thread is a function that is executed in concurrency with the main thread.

In order to exploit thread functionalities, library \texttt{pthreads.h} must be included and the C program must be compiled with the option \texttt{lpthread}.

Pthreads library defines more than 60 functions. The most important and used are:
\begin{description}
\item \texttt{pthread\_t pthread\_self()} returns the thread identifier of the calling thread;
\item \texttt{int pthread\_create(pthread\_t *tid, const pthread\_attr\_t *attr, \newline void *(*startRoutine)(void *), void *arg)} allows creating a new thread executing the C function \texttt{startRoutine};
\item \texttt{void pthread\_exit(void *valuePtr)} allows a thread to terminate returning a termination status. This value is kept by the kernel until a thread calls \texttt{pthread\_join};
\item \texttt{int pthread\_join(pthread\_t tid, void **valuePtr)} is used by a thread to wait the termination of another thread, if the thread was declared as \emph{joinable};
\item \texttt{int pthread\_detach(pthread\_t tid)} declares the thread \texttt{tid} as detached. Therefore, its status information will not be kept by the kernel at the termination and no thread can join with that thread;
\item \texttt{int pthread\_cancel(pthread\_t tid)} terminates the target thread. It is a dangerous operation because the calling thread does not wait for the termination of target thread, so it is not possible to know the status of the terminated thread.
\end{description}

\paragraph{Thread termination}
A process, with all its threads, terminates if
\begin{itemize}
\item Its thread calls \texttt{exit};
\item The main thread execute \texttt{return} which is compiled as an \texttt{exit};
\item The main thread receives a signal whose action is to terminate.
\end{itemize}
A single thread can terminate without affecting the other process threads
\begin{itemize}
\item Executing \texttt{return};
\item Executing \texttt{pthread\_exit};
\item Receiving a cancellation request performed by another thread.
\end{itemize}

The kernel does not distinguish which
function executed the \texttt{exit} statement, so whichever thread calls \texttt{exit} will cause the entire process termination.
If the main wants to stop and keep alive
other threads, it must perform a \texttt{pthread\_exit} rather than a \texttt{return}.