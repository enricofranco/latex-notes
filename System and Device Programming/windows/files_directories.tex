\chapter{File and directory}
\section{File and directory management}
\subsection{Delete a file}
Delete a file by specifying a file name. It does not work with an open file or with a directory.

\begin{verbatim}
BOOL DeleteFile(
  LPCTSTR lpFileName
);
\end{verbatim}

\paragraph{Returned value}
\begin{itemize}
\item \texttt{TRUE} if and only if the delete operation succeeds;
\item \texttt{FALSE} otherwise.
\end{itemize}

\paragraph{Parameters}
\begin{description}
\item [\texttt{lpFileName}] relative or absolute pathname.
\end{description}

\subsection{Rename or move a file}
\texttt{MoveFileEx} is the extended version, not supported by earlier Windows versions. Use \texttt{MoveFileEx} to overwrite existing files or move files on different drivers, implemented on a copy and then a delete operation.

\begin{Parallel}{}{}
\ParallelLText{
\begin{verbatim}
BOOL MoveFile (
  LPCTSTR lpExisting,
  LPCTSTR lpNew
);
\end{verbatim}}
\ParallelRText{
\begin{verbatim}
BOOL MoveFile (
  LPCTSTR lpExisting,
  LPCTSTR lpNew,
  DWORD dwFlags
);
\end{verbatim}}
\end{Parallel}

\paragraph{Returned value}
\begin{itemize}
\item \texttt{TRUE} if it succeeds;
\item \texttt{FALSE} if it fails.
\end{itemize}

\paragraph{Parameters}
\begin{description}
\item [\texttt{lpExisting}] the name of the existing file or directory.
\item [\texttt{lpNew}] the name of the new file or directory. In general:
\begin{itemize}
\item Wildcards are not allowed in files or directories names;
\item Directory must be on the same drive;
\item If \texttt{lpNew} is \texttt{NULL} the file or directory is deleted.
\end{itemize}
\item [\texttt{dwFlags}]:
\begin{description}
\item [\texttt{MOVEFILE\_REPLACE\_EXISTING}] Replace an existing file.
\item [\texttt{MOVEFILE\_WRITE\_THROUGH}] Do not return until the copied file is flushed through to the disk.
\item [\texttt{MOVEFILE\_COPY\_ALLOWS}] Need to be used to copy on a new volume. When copying to a different volume, copy then delete old file.
\item [\texttt{MOVEFILE\_DELAY\_UNTIL\_REBOOT}] Administration: copy when restart the OS.
\end{description}
\end{description}

\section{Directory management}
\subsection{Create directory}
Create a new empty directory.

\begin{verbatim}
BOOL CreateDirectory (
  LPCTSTR lpPath,
  LPSECURITY_ATTRIBUTES lpsa
);
\end{verbatim}

\paragraph{Returned value}
\begin{itemize}
\item \texttt{TRUE} if it succeeds;
\item \texttt{FALSE} if it fails.
\end{itemize}

\paragraph{Parameters}
\begin{description}
\item [\texttt{lpPath}] points to a \texttt{NULL}-terminated string with the directory name to be created.
\item [\texttt{lpsa}] security attributes, \texttt{NULL} for now.
\end{description}

\subsection{Remove directory}
Remove a directory, which must be empty.

\begin{verbatim}
BOOL RemoveDirectory (
  LPCTSTR lpPath
);
\end{verbatim}

\paragraph{Returned value}
\begin{itemize}
\item \texttt{TRUE} if it succeeds;
\item \texttt{FALSE} if it fails.
\end{itemize}

\paragraph{Parameters}
\begin{description}
\item [\texttt{lpPath}] points to a \texttt{NULL}-terminated string with the directory name to be removed.
\end{description}

\subsection{Set current directory}
Each \emph{process} has a current working directory. There is a current directory for each drive. Programs can set and get the current directory.

The current directory is \textbf{global} to a process and is shared by all threads in a process. Hence, concurrent threads cannot set different directories, they have to use absolute paths.

\begin{verbatim}
BOOL SetCurrentDirectory (
  LPCTSTR lpCurDir
);
\end{verbatim}

\paragraph{Returned value}
\begin{itemize}
\item \texttt{TRUE} if it succeeds;
\item \texttt{FALSE} if it fails.
\end{itemize}

\paragraph{Parameters}
\begin{description}
\item [\texttt{lpCurDir}] the path to the new current directory, it can be a relative or an absolute path.
\end{description}

\subsection{Get current directory}
It gets the full current pathname, and it returns it into the specified buffer.

\begin{verbatim}
BOOL FetCurrentDirectory (
  DWORD cchCurDir,
  LPTSTR lpCurDir
);
\end{verbatim}

\paragraph{Returned value}
\begin{itemize}
\item The string length of the returned pathname;
\item The required buffer size if the buffer is not large enough, this includes the space for the \texttt{NULL} string terminator;
\item Zero if the function fails.
\end{itemize}

\paragraph{Parameters}
\begin{description}
\item [\texttt{cchCurDir}] character length of the buffer for the directory name. Windows uses this technique whenever the result's length is not known.
\item [\texttt{lpCurDir}] points to the buffer to receive the pathname string. Be sure the buffer is really as long as it is declared to be. Potential buffer overflow.
\end{description}

\section{File and directory processing}
Reading a directory content can be done following the same logic used to read a file. It is required to:
\begin{enumerate}
\item Open the directory, i.e.\@ generate a search handle satisfying specific requirements;
\item Read the directory content;
\item End the reading operation, i.e.\@ close the directory.
\end{enumerate}

\subsection{File searching}
\subsubsection{First file}
A file search requires a \emph{search handle}.
\begin{verbatim}
HANDLE FindFirstFile (
  LPCTSTR lpSearchFile,
  LPWIN32_FIND_DATA lpffd
);
\end{verbatim}
\texttt{FindFirstFile} examines all entries of the current directory, looking for a file match with \texttt{lpSearchFile}. It evaluates a search handle required for subsequent operation and a pointer to a data structure containing information on the entry itself.

\paragraph{Returned value}
\begin{itemize}
\item A ``search handle'': the handle can be used to obtain information on the current entry;
\item The constant value \texttt{INVALID\_HANDLE\_VALUE} when has been a failure.
\end{itemize}

\paragraph{Parameters}
\begin{description}
\item [\texttt{lpSearchFile}] points to directory or pathname. Wildcards can be used (\texttt{*} and \texttt{?}).
\item [\texttt{lpffd}] points to a \texttt{WIN32\_FIND\_DATA} structure. The structure contains information on the entry.
\end{description}

\subsubsection{Next files}
Once the \texttt{HANDLE} given by \texttt{FindFirstFile} is available, \texttt{FindNextFile} may obtain the data info for subsequent entry.

\begin{verbatim}
BOOL FindNextFile (
  HANDLE hFindFile,
  LPWIN32_FIND_DATA lpffd
);
\end{verbatim}

\paragraph{Returned value}
\begin{itemize}
\item \texttt{TRUE} when the search can go on.
\item \texttt{FALSE} when no more files satisfy the search pattern.
\end{itemize}

\subsubsection{Close search}
When the search is complete, close the search handle. The directory \texttt{HANDLE} is not closed with \texttt{CloseHandle}.
\begin{verbatim}
BOOL FindClose (
  HANDLE hFindFile
);
\end{verbatim}

\subsection{WIN32\_FIND\_DATA Structure}
\begin{verbatim}
typedef struct _WIN32_FIND_DATA {
  DWORD dwFileAttributes;
  FILETIME ftCreationTime;
  FILETIME ftLastAccessTime;
  FILETIME ftLastWriteTime;
  DWORD nFileSizeHigh;
  DWORD nFileSizeLow;
  DWORD dwReserved0;
  DWORD dwReserved1;
  TCHAR cFileName[MAX_PATH];
  TCHAR cAlternateFileName[14];
} WIN32_FIND_DATA;
\end{verbatim}

\begin{description}
\item [\texttt{dwFileAttributes}] see \texttt{CreateFile} attributes.
\item [\texttt{ftCreationTime}] 64-bit integer.
\item [\texttt{ftLastAccessTime}] 64-bit integer.
\item [\texttt{ftLastWriteTime}] 64-bit integer.
\item [\texttt{dwReserved0}] reserved for future use.
\item [\texttt{dwReserved1}] reserved for future use.
\item [\texttt{cFileName}] filename.
\item [\texttt{cAlternateFileName}] MS-DOS filename. Limited to 8 characters for filename and 3 characters for extension.
\end{description}