\chapter{Asynchronous Input/Output}
\section{Overview}
Input and output are inherently slow compared to other processing due to delays caused by track and sector seek time on random access devices or by relatively slow data transfer rate between physical device and system memory.

\paragraph{Windows asynchronous I/O}
\begin{itemize}
\item Threads and processes: each thread within a process performs normal synchronous I/O but other threads can continue execution;
\item Overlapped I/O: a thread continues execution after issuing a read, write, or other I/O operation. The thread then waits on either the handle or a specified event when it requires the I/O result before continuing;
\item Completion routines: the system invokes a specified \emph{completion routine} within the thread when the I/O operation completes. They are also called extended I/O or alertable I/O.
\end{itemize}

\section{Overlapped I/O}
Overlapped structures are options on four I/O functions that can potentially block while the operation completes: \texttt{ReadFile}, \texttt{WriteFile}, \texttt{ConnectNamedPipe} and \texttt{TransactNamedPipe}.

It is needed to specify \texttt{FILE\_FLAG\_OVERLAPPED} as part of \texttt{fdwAttrsAndFlags} for \texttt{CreateFile}, or as part of \texttt{fdwOpenMode} for \texttt{CreateNamedPipe}. Overlapped I/O can be used with Windows Sockets, too.

\paragraph{Consequences}
I/O operations do not block. The system returns immediately from \texttt{ReadFile}, \texttt{WriteFile}, \texttt{TransactNamedPipe} and \texttt{ConnectNamedPipe}. The returned function value and the returned number of bytes transferred are not useful to indicate success or failure because the I/O operation is most likely not yet completed

The program may issue multiple reads or writes on a single handle, therefore the file pointer is meaningless. The program must be able to wait on I/O completion: in case of multiple outstanding operations on a single handle, it must be able to determine which operation completed because I/O operations do not necessarily complete in the same order as they were issued.

\paragraph{Overlapped structure}
\begin{verbatim}
typedef struct _OVERLAPPED {
  DWORD Internal;
  DWORD InternalHigh;
  DWORD OffsetLow;
  DWORD OffsetHigh;
  HANDLE hEvent;
} OVERLAPPED;
\end{verbatim}
The overlapped structure, indicates the file position on 64 bits, the event that will be signaled when the operation completes, specified, for example, by the \texttt{lpOverlapped} parameter of \texttt{ReadFile}. There are also two internal \texttt{DWORD}s that the programmer does not use.

\begin{itemize}
\item File position must be set in \texttt{Offset} and \texttt{OffsetHigh};
\item \texttt{hEvent} is an event \texttt{HANDLE} created with \texttt{CreateEvent}. The event can be named or unnamed, but it must be a \emph{manually reset event}; if \texttt{hEvent} is \texttt{NULL}, wait is performed on the file handle.
\end{itemize}
The event is immediately reset by the system when the program makes an I/O call and, when it completes, the event is signaled.

The overlapped structure is a possible alternative to \texttt{SetFilePointer} even if the file handle is synchronous.

\paragraph{Cautions}
\begin{itemize}
\item Do not reuse an \texttt{OVERLAPPED} structure while its associated I/O operation is outstanding;
\item Do not reuse an event while it is part of an overlapped structure;
\item In case of more than one outstanding request on an overlapped handle, use events for synchronization rather than the file handle.
\end{itemize}

\paragraph{Overlapped I/O states}
An overlapped \texttt{ReadFile} or \texttt{WriteFile} operation returns immediately. In most cases, the I/O will not be complete, the operation returns a \texttt{FALSE} and \texttt{GetLastError} will return \texttt{ERROR\_IO\_PENDING}.

\texttt{GetOverlappedResult} allows to determine how many bytes were actually transferred.

\begin{verbatim}
BOOL GetOverlappedResult (
  HANDLE hFile,
  LPOVERLAPPED lpoOverlapped,
  LPWORD lpcbTransfer,
  BOOL fWait
);
\end{verbatim}

\paragraph{Returned value}
\begin{itemize}
\item \texttt{TRUE} only if operations has completed;
\item \texttt{FALSE} otherwise.
\end{itemize}

\paragraph{Parameters}
\begin{description}
\item [\texttt{hFile}] Handle to the file, named pipe, or communications device. This is the same handle that was specified when the overlapped operation was started.
\item [\texttt{lpoOverlapped}] Pointer to an \texttt{OVERLAPPED} structure.
\item [\texttt{lpcbTransfer}] Pointer to a variable that receives the number of bytes that were actually transferred by a read or write operation.
\item [\texttt{fWait}] If \texttt{TRUE} specifies that the function will wait until specified operation completes. Otherwise, it returns immediately.
\end{description}

\section{Extended I/O}
Rather than requiring a thread to wait for a completion signal on an event or handle, the system can invoke a user-specified \emph{completion routine} when an I/O operation completes.

\subsubsection{Extended read and write}
\begin{verbatim}
BOOL ReadFileEx (
  HANDLE hFile,
  LPVOID lpBuffer,
  DWORD nNumberOfBytesToRead,
  LPOVERLAPPED lpOverlapped,
  LPOVERLAPPED_COMPLETION_ROUTINE lpcr
);
\end{verbatim}

\begin{verbatim}
BOOL WriteFileEx (
  HANDLE hFile,
  LPVOID lpBuffer,
  DWORD nNumberOfBytesToWrite,
  LPOVERLAPPED lpOverlapped,
  LPOVERLAPPED_COMPLETION_ROUTINE lpcr
);
\end{verbatim}

The extended read and write functions can be used with open file and named pipe handles if \texttt{FILE\_FLAG\_OVERLAPPED} was used at open or create time. The two functions are familiar but have an extra parameter to specify the completion routine. The overlapped structure must be supplied without need to specify the \texttt{hEvent} member.

The extended functions do not require the parameters for the number of bytes transferred because this information is conveyed to the completion routine, which must be included in the program. In fact, completion routine has parameters for the byte count, an error code, and the overlapped structure.

\begin{verbatim}
VOID WINAPI FileIOCompletionRoutine (
  DWORD fdwError,
  DWORD cbTransferred,
  LPOVERLAPPED lpo
);
\end{verbatim}
\texttt{FileIoCompletionRoutine} is a place holder, not an actual function name. \texttt{fdwError} is limited to \texttt{0}, meaning success.

Two things must happen before the completion routine is invoked by the system:
\begin{itemize}
\item The actual I/O operation must complete;
\item The calling thread must be in an \emph{alertable wait} state, i.e.,\@ it must make an explicit call to one of three alertable wait functions.
\end{itemize}

\subsubsection{Alertable wait functions}
\begin{verbatim}
DWORD WaitForSingleObjectEx (
  HANDLE hObject,
  DWORD dwTimeOut,
  BOOL fAlertable
);
\end{verbatim}

\begin{verbatim}
DWORD WaitForMultipleObjectsEx (
  DWORD cObjects,
  LPHANDLE lphObjects,
  BOOL fWaitAll,
  DWORD dwTimeOut,
  BOOL fAlertable
);
\end{verbatim}

\begin{verbatim}
DWORD SleepEx (
  DWORD dwTimeOut,
  BOOL fAlertable
);
\end{verbatim}

Each function has an \texttt{fAlertable} flag which must be set to \texttt{TRUE}. These three functions will return as soon as one of the following occurs:
\begin{itemize}
\item Handle(s) are signaled so as to satisfy one of the first two functions;
\item Any completion routine in the thread finishes;
\item The time period expires (it could also be \texttt{infinite}).
\end{itemize}
There are no events associated with the \texttt{ReadFileEx} and \texttt{WriteFileEx} overlapped structures. \texttt{SleepEx} is not associated with a synchronization object.

\paragraph{Execution of completion routines}
As soon as an extended I/O operation is complete, its associated completion routine is queued for execution. All of a thread's queued completion routines will be executed when the thread enters an alertable wait state, and the alertable wait function returns only after the completion routines return.

\texttt{SleepEx} will return \texttt{WAIT\_IO\_COMPLETION} if one or more queued completion routines were executed. \texttt{GetLastError} will return this same value after one of the wait function returns. It is possible to use a 0 timeout value with any alertable wait function, and it will return immediately.

It is possible to use \texttt{hEvent} data member of the overlapped structure to convey any information to the completion routine.