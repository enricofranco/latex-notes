\chapter{Configuration management}
A software is made of many parts (i.e.,\@ documents, programs, etc.) with different instances (customer, platforms, etc.) and thousands of separate documents are generated for a large software system. A software system changes because of different instances of software for different customers and because the same software changes over time.

Therefore, it is needed to identify the history of a document (versioning), the correct set of documents for a specific need (configuration), who can access and change what (change control) and how the system is obtained (build). \emph{Configuration management} permits to identifiy and manage parts of software, control access and changes to parts and allow to rebuild previous version of software.

\section{Versioning}
\emph{Versioning} permits to keep track of versions using same names or different version names in such a way that a user decides when to change version (commit operation) and that is always possible to recover a past version.

\paragraph{Configuration item}
A \emph{configuration item} (CI) is an element under configuration control. It is characterized by a name and a version number and all its version numbers are kept. The user decides to change version number with specific operation (\textbf{commit}) and it is possible to retrieve any version.

It is the basic unit for the configuration management system, e.g.,\@ a work product or piece of software that is treated as a single entity for the purpose of configuration management. It may correspond to one or more documents or one or more programs.

Not all documents in a project become a configuration item. If all documents are considered configuration items, a large overhead is introduced but, on the other hand, if no document is considered configuration item, no history is available and there is no information on configurations.

\paragraph{Configuration}
Configuration items have depencencies among them, some a re syntactically defined (e.g.,\@ \texttt{\#include} or \texttt{import} statements) but most are not, such as requirement and test cases. This requires a way to treat sets of configuration items, i.e.,\@ \emph{traceability}.

\begin{figure}[hbtp]
\centering
\includegraphics[scale=0.3]{images/dependency_traceability.png}
\caption{Dependency and traceability}
\end{figure}

A \emph{configuration} is defined as a set of items, each in a specific version and, generally, it includes not only source code. The same configuration item may appear in different configurations. Therefore, also a configuration has its version. There are two main approaches to identify a configuration:
\begin{enumerate}
\item ID for configuration, ID for configuration item: CVS;
\item ID for configuration, same ID for all configuration items in the configuration: Subversion, Git.
\end{enumerate}

\subparagraph{Approach 1}
Configuration items numbering and configuration numbering are independent. In this way, it is clear when a configuration item changes but the name of configuration item and configuration are managed separately.

\begin{figure}[hbtp]
\centering
\includegraphics[scale=0.3]{images/configuration_approach1.png}
\caption{Configuration - Approach 1}
\end{figure}

\subparagraph{Approach 2}
Configuration number changes on every change. Items inherits the configuration numbering. In this way, configuration items and configuration are managed together, therefore it is unclear when a configuration item changes.

\begin{figure}[hbtp]
\centering
\includegraphics[scale=0.3]{images/configuration_approach2.png}
\caption{Configuration - Approach 2}
\end{figure}

\paragraph{Version}
A \emph{version} is a particular instance of a configuration item at a certain point in time. Each instance as a unique number and each configuration has a unique number.

\paragraph{Baseline}
A \emph{baseline} identifies a configuration in a stable, frozen form. Not all configurations are baselines.

\section{Change control}
Typically, a team develops a software. Hence, many people need to access parts of software contained in a common repository (shared folder) in such a way that all can read and write documents and programs.

\begin{description}
\item [Repository] Place where configuration items and configurations are stored.
\item [Workspace] Place where users work, on copies of configuration items.
\end{description}

\begin{figure}[hbtp]
\centering
\includegraphics[scale=0.3]{images/repository_workspace.png}
\caption{Repository and workspaces}
\end{figure}

\subsection{Repository}
A \emph{repository} can implemented as a shared folder, containing shared files, on file server or using a CMS\footnote{Content Management System.} tool with change control (check-in, check-out). Changes must be disciplined, requiring knowledge about who controls, what is controlled and how control is implemented.

A repository can be implemented exploiting different approaches:
\begin{itemize}
\item Local: repository and user on the same machine;
\item Centralized: repository on a single server, many users (clients) can access and obtain copies of files;
\item Distributed: repository mirrored on all servers/clients.
\end{itemize}

\begin{description}
\item [Check-out] Extraction of configuration items from repository with purpose of changing it or not. After checkout. next users are notified.
\item [Check-in] Insertion of configuration items under control.
\end{description}

\begin{center}
\begin{tabularx}{\textwidth}{X|X}
\textbf{Check in/out} & \textbf{File system} \\
\hline
Configuration items are in repository & Files are in shared directory \\
\hline
To read/write a configuration item, user needs to perform checkout & Any user can get copy of file, or work on original \\
\hline
After checkout, next user knows that configuration item is used by someone else & Users can work on copies of file without knowing that others are doing the same
\end{tabularx}
\end{center}

\paragraph{Scenarios}
\subparagraph{Lock modify unlock - Serialization}
One can change at a time. Thus, no parallel work and delays. Problems if locker forgets to unlock. As shown in figure~\ref{img:lock_modify_unlock}, first check out locks the file and no other checkouts are allowed until checkin.

\begin{figure}[hbtp]
\centering
\includegraphics[scale=0.4]{images/lock_modify_unlock.png}
\caption{Lock modify unlock}
\label{img:lock_modify_unlock}
\end{figure}

\subparagraph{Copy modify merge}
Many change in parallel, the merge. It is more flexible but requires care to resolve possible conflicts. As shown in figure~\ref{img:copy_modify_merge}, the second check in signals conflict. User has to do manual merge of \emph{x+1} and \emph{x+1'} in \emph{x+2}.

\begin{figure}[hbtp]
\centering
\includegraphics[scale=0.4]{images/copy_modify_merge.png}
\caption{Copy modify merge}
\label{img:copy_modify_merge}
\end{figure}

\paragraph{CMS approaches}
\begin{itemize}
\item Local: a simple database that kept all the changes to files under revision control, e.g.,\@ RCS;
\item Centralized: a single server that contains all the versioned files, and a number of clients that check out files from that central place, e.g.,\@ CSV, Subversion;
\item Distributed: clients fully mirror the repository, e.g.,\@ Git, Mercurial.
\end{itemize}

Very often, it is needed to proceed with \emph{parallel versions}, while keeping trace of common origin, as shown in figure~\ref{img:branches_merges}. For example, when some developers work on branch to fix defect, while other work to add functionalities.

\begin{figure}[hbtp]
\centering
\includegraphics[scale=0.5]{images/branches_merges.png}
\caption{Branches}
\label{img:branches_merges}
\end{figure}

\paragraph{CM planning}
\emph{CM plan} contains key CM related choices and policies in a project, i.e.,\@ requirement and design documents, source files, test plan, test cases, configuration files, parameters, tools used in the production chain. It is needed to define which CM tool to use, what should and should not be a configuration item, control policies and who is the CM manager.

\paragraph{Build}
System \emph{building} is the process of compiling and linking software components into an executable system. Different systems are built from different combinations of components.

Systems are normally represented for building by specifying the filename to be processed by building tools. Dependencies between these are described to the building tools (e.g.,\@ Make, Ant, Maven). Mistakes can be made as users lose track of which objects are stored in which files. A system modelling language addresses this problem by using a logical rather than a physical system representation.

\begin{figure}[hbtp]
\centering
\includegraphics[scale=0.4]{images/dependencies.png}
\caption{Example of dependencies}
\end{figure}

\section{Git}
\subsection{Configuration management system}
A \emph{configuration management system} is a system that records changes to a file or set of files over time so that specific versions can be recalled later. Its functionalities are:
\begin{itemize}
\item Revert files back to a previous state;
\item Revert the entire project back to a previous state;
\item Compare changes over time;
\item Monitor who last modified something that might causing a problem;
\item Monitor who introduced an issue and when.
\end{itemize}

\paragraph{Data management models}
\begin{description}
\item [\textbf{Differences}] Information is kept as a set of files and the changes made to each file over time.
\item [\textbf{Snapshots}] Every commit, a picture of what all the files look like at a moment, is taken and the system stores a reference to that snapshot. If files have not changed, system stores just a link to the previous identical file.
\end{description}

\subsection{Introduction}
Git is a distributed CMS that uses snapshots. It exploits local operation and provides integrity. In fact, everything is check-summed before it is stored and is referred to by that checksum. No untracked changes of any file or directory are possible. Git only adds data. Each file has tree states:
\begin{itemize}
\item \textbf{Committed}: data is safely stored in local database;
\item \textbf{Modified}: the file has changed but it is not committed to the database yet;
\item \textbf{Staged}: a modified file is marked in its current version to go into next commit.
\end{itemize}

\begin{figure}[hbtp]
\centering
\includegraphics[scale=0.4]{images/cms_distribution.png}
\caption{CMS distribution}
\end{figure}

\paragraph{Sections}
\begin{description}
\item [\textbf{Git directory}] It stores the metadata and object database for the project. It is what is copied when a repository is cloned from another computer.
\item [\textbf{Working directory}] It is a single checkout of one version of the project. These files are pulled out of the compressed database in the Git directory and placed on disk to use or modify.
\item [\textbf{Staging area}] It stores information on what will go into next commit. Sometimes referred to as the \emph{index}.
\end{description}

\paragraph{Typical workflow}
\begin{enumerate}
\item Modify files in working directory;
\item Stage the files, adding snapshots of them to the staging area;
\item Do a commit, which takes the files as they are in the staging area and stores that snapshots permanently to the Git directory.
\end{enumerate}

\begin{figure}[hbtp]
\centering
\includegraphics[scale=0.4]{images/file_lifecycle.png}
\caption{Lifecycle of files}
\end{figure}

\emph{Github} is an online code hosting service built on top of Git, mainly used by OSS projects but commercial use is growing. Nowadays, it is used by PayPal, SAP, vimeo and others.

\subsection{Basic commands}
\paragraph{Recording changes}
\begin{description}
\item [\texttt{git status}] determine which files are in which state.
\item [\texttt{git add}] track a new file.
\item [\texttt{git diff}] know exactly what is changed.
\item [\texttt{git commit}] commit changes.
\item [\texttt{git rm}] remove a file from the tracked files and also from the working directory.
\item [\texttt{git mv}] rename a file. Git doesn't explicitly track file movement. If a file is renamed in Git, no metadata is stored in it that tells the file has been renamed.
\end{description}

\paragraph{Commit history}
\begin{description}
\item [\texttt{git log}] look back to see what has happened in a repository.
\end{description}

\paragraph{Working with remotes}
\begin{description}
\item [\texttt{git pull}] automatically fetch and then merge a remote branch into your current branch.
\item [\texttt{git push}] share changes, i.e.,\@ transfer them on the server.
\end{description}

\subsection{Branching}
Git stores a commit object that contains a pointer to the snapshot of the context staged and pointers to the commit or commits that directly cam before this commits, i.e.,\@ its parent or parents. Staging the files, Git checksums each one, stores that version of the file (blobs) and adds that checksum to the staging area.

\paragraph{Branches}
A branch in Git is simply a lightweight movable pointer to commits. The default branch name in Git is \texttt{master}. Every commit, it moves forward automatically. \texttt{HEAD} pointer points to the current local branch.

Figure~\ref{img:git_branches} shows the status of commands:
\begin{enumerate}
\item \texttt{git branch testing} which creates a branch named \texttt{testing} but does not switch to that branch;
\item \texttt{git checkout testing} moves the \texttt{HEAD} pointer to that branch;
\item \texttt{git commit}.
\end{enumerate}

\begin{figure}[hbtp]
\centering
\includegraphics[scale=0.4]{images/git_branches.png}
\caption{Git branches}
\label{img:git_branches}
\end{figure}

\subparagraph{Switching branches}
Switching branches revert the files in working directory back to the snapshot that \texttt{master} points to. The changes made from this point forward will diverge from an older version of the project. Files in the working directory changes.

\subparagraph{Merging branches}
Figure~\ref{img:git_branches_merges} shows the status of commands:
\begin{enumerate}
\item \texttt{git merge hostfix} merges \texttt{hostfix} with \texttt{master}, assuming that current branch is \texttt{hostfix};
\item \texttt{git commit} on \texttt{iss53};
\item \texttt{git merge iss53} merges \texttt{iss53} with \texttt{master}.
\end{enumerate}
\begin{figure}[hbtp]

\centering
\includegraphics[scale=0.4]{images/git_branches_merges.png}
\caption{Git merge}
\label{img:git_branches_merges}
\end{figure}

Instead of just moving the branch pointer forward, Git creates new snapshot that results from this three-way merge and automatically creates a new commit that points to it. This is referred to as \emph{merge commit}, and is special in that is has more than one parent.

Git determines the best common ancestor to use for its merge base. Conflicts are possible, in this case merging is not done until conflicts are managed, as for commits.