\chapter{User Interface design}
\emph{User interface design} is more important for consumer and enterprise market with respect to embedded market. 

User interface is the portion of the system interacting with the user and, nowadays, three user interfaces are possible: web, desktop and mobile.

% Complexity of usage, Complexity of structure, Complexity of functions

\paragraph{Principles and ideas}
\begin{description}
\item [Ergonomic] Safety, adaptability, comfort, usability, \dots
\item [Emotional design] Beyond ergonomics, the interaction should cause positive emotions in the user;
\item [User experience] Usability + feelings + emotions + value;
\item [Transparent technology] No emphasis on technology, emphasis on interaction;
\item [Feedback, user centered design] No decision based on personal opinions, but feedback from real users.
\end{description}

\section{User centered design}
\begin{enumerate}
\item Identify the users
\begin{description}
\item [Personas] Identify and describe typical user representative of a class of users. For each persona, describe life scenarios and for each persona or scenario, identify possible interaction with application or object.
\end{description}

\item Define requirements
\begin{description}
\item [Focus group] A moderator starts and monitors discussion of a group of homogeneous people on a defined topics. The discussion can be open or script guided. This technique is dynamic and very productive but long.
\item [Questionnaires] Written questions with open or close answers. This technique provides a strict and statistical analysis which covers more data points.
\item [Interviews] Deep discussion, open or guided by script or questions, one to one reporting log. This technique provides a detailed analysis but requires a very long time.
\item [Ethnographics] Researcher is hidden in an environment and observes facts and behavior of the user and the population. This technique is expensive, long and risk of being invasive must be considered.
\end{description}

\item Define system and interactions
\begin{description}
\item [Prototypes] Low fidelity (sketches, post its) or high fidelity (computer executable mock ups). 
\end{description}

\item Feedback
\begin{description}
\item [A/B test] Try some variations of the user interface on a group of users and measures are taken, i.e.,\@ how many users achieve the goal. 
\end{description}
\end{enumerate}