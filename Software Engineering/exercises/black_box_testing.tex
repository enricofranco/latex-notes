\chapter{Black box testing}
\section{Int converter}
A function converts a sequence of chars in an integer number. The sequence can start with a `-' (negative number). If the sequence is shorter than 6 chars, it is filled with blanks (to the left side).

The integer number must be in the range minint = -32768 to maxint = 32767. The function signals an error if the sequence of chars is not allowed. The sequence of chars must be $\le 6$ chars.

\begin{itemize}
\item Define equivalence classes and related boundary conditions;
\item For each class, define at least one test case;
\item Define class Cartesian product. 
\end{itemize}
\texttt{int convert(string s);}

\subsection{Solution}
\paragraph{Criteria}
\begin{itemize}
\item Sign of the number: positive, negative;
\item Number length: $< 6$, $== 6$, $> 6$;
\item Well formed integer number: yes, no;
\item In range -32768, 32767: yes, no.
\end{itemize}
Actually, it is needed to combine different criteria. Table~\ref{tab:int_converter_first} shows some test cases which are not good enough, in fact some test cases, the ones highlighted, can belong to different classes and combinations have to be tested, too.

Table~\ref{tab:int_converter_second} shows how to perform criteria combinations. It is important to choose a ``good'' order for the criteria in order to prune as early as possible the number of combinations.

\begin{table}
\centering
\small
\begin{tabular}{|l|l|l|}
\hline
\textbf{Criterion} & \textbf{Valid} &\textbf{Invalid} \\
\hline
\multirow{4}{*}{Sign} & Positive & \\
&	{
	\begin{tabular}{l} % Positive
	T1(``1234''; 1234) \\
	\end{tabular}
	} & \\
\cline{2-2} % split positive and negative
& Negative & \\
&	{
	\begin{tabular}{l} % Negative
	T2(``-1234''; -1234) \\ 
	\end{tabular}
	} & \\
\hline % Sign end
\multirow{5}{*}{Length} & $< 6$ & $> 6$ \\
& 	{
	\begin{tabular}{l} % < 6
	T3(``12345''; 12345) \\
	\end{tabular}
	}
&	\multirow{4}{*}{ % > 6
	\begin{tabular}{l}
	\colorbox{yellow}{T6(``1234567''; error)} \\
	T7(``0004567''; error) \\
	\end{tabular}
	} \\
\cline{2-2} % split > 6 and == 6
& $== 6$ & \\
&	{
	\begin{tabular}{l} % = 6
	\colorbox{yellow}{T4(``123456''; error)} \\
	T5(``-12345''; -12345) \\
	\end{tabular}
	} & \\
\hline % Lenght end
\multirow{3}{*}{Well formed integer} & Y & N \\
&	{
	\begin{tabular}{l} % Yes
	T8(``234''; 234) \\
	T1 T2 T3 T5 \\
	\end{tabular}
	}
&	{
	\begin{tabular}{l} % No
	T9(``12ff99''; error) \\
	T10(``ff''; error) \\
	\end{tabular}
	} \\
\hline % Well formed end
\multirow{5}{*}{Range} & -32768, 32767 & > 32767 \\ 
&	\multirow{3}{*}{
	\begin{tabular}{l} % In range
	T1, T2, T3, T5, T8
	\end{tabular}
	}
&	{
	\begin{tabular}{l} % Not in range, greater
	T5 \\
	T11(``40000''; error)
	\end{tabular}
	} \\
\cline{3-3} % split out of range
& & < -32768 \\
& &	{
	\begin{tabular}{l}
	T12(``-40000''; error)
	\end{tabular}
	} \\
\hline
\end{tabular}
\caption{Int converter - Test cases}
\label{tab:int_converter_first}
\end{table}

\begin{table}
\centering
\small
\begin{tabular}{|l|l|l|l|l|}
\hline
\textbf{Sign} & \textbf{Length} & \textbf{Well formed} & \textbf{Range} & \textbf{Test cases} \\
\hline
\multirow{18}{*}{Positive} & \multirow{6}{*}{$< 6$} & \multirow{3}{*}{Yes} & -32768, 32767 & T1(``1234''; 1234) \\
\cline{4-5}
& & & > 32767 & T11(``40000''; error) \\
\cline{4-5}
& & & < -32768 & Unfeasible \\
\cline{3-5}
& & \multirow{3}{*}{No} & -32768, 32767 & \multirow{3}{*}{No sense} \\
\cline{4-4}
& & & > 32767 & \\
\cline{4-4}
& & & < -32768 & \\
\cline{2-5}
& \multirow{6}{*}{$> 6$} & \multirow{3}{*}{Yes} & -32768, 32767 & T7(``0004567''; error) \\
\cline{4-5}
& & & > 32767 & T6(``1234567''; error) \\
\cline{4-5}
& & & < -32768 & Unfeasible \\
\cline{3-5}
& & \multirow{3}{*}{No} & -32768, 32767 &  \multirow{3}{*}{No sense} \\
\cline{4-4}
& & & > 32767 & \\
\cline{4-4}
& & & < -32768 & \\
\cline{2-5}
& \multirow{6}{*}{$== 6$} & \multirow{3}{*}{Yes} & -32768, 32767 & T13(``012345''; 12345) \\
\cline{4-5}
& & & > 32767 & T14(``123456''; error) \\
\cline{4-5}
& & & < -32768 & Unfeasible \\
\cline{3-5}
& & \multirow{3}{*}{No} & -32768, 32767 & \multirow{3}{*}{No sense} \\
\cline{4-4}
& & & > 32767 & \\
\cline{4-4}
& & & < -32768 & \\
\cline{1-5}
\multirow{18}{*}{Negative} & \multirow{6}{*}{$< 6$} & \multirow{3}{*}{Yes} & -32768, 32767 & T2(``-1234''; -1234) \\
\cline{4-5}
& & & > 32767 & Unfeasible \\
\cline{4-5}
& & & < -32768 & Unfeasible \\
\cline{3-5}
& & \multirow{3}{*}{No} & -32768, 32767 & \multirow{3}{*}{No sense} \\
\cline{4-4}
& & & > 32767 & \\
\cline{4-4}
& & & < -32768 & \\
\cline{2-5}
& \multirow{6}{*}{$> 6$} & \multirow{3}{*}{Yes} & -32768, 32767 & T15(``-030000''; error) \\
\cline{4-5}
& & & > 32767 & Unfeasible \\
\cline{4-5}
& & & < -32768 & T16(``-400000''; error)  \\
\cline{3-5}
& & \multirow{3}{*}{No} & -32768, 32767 & \multirow{3}{*}{No sense} \\
\cline{4-4}
& & & > 32767 & \\
\cline{4-4}
& & & < -32768 & \\
\cline{2-5}
& \multirow{6}{*}{$== 6$} & \multirow{3}{*}{Yes} & -32768, 32767 & T5(``-12345''; -12345) \\
\cline{4-5}
& & & > 32767 & Unfeasible \\
\cline{4-5}
& & & < -32768 & T12(``-40000''; error) \\
\cline{3-5}
& & \multirow{3}{*}{No} & -32768, 32767 & \multirow{3}{*}{No sense} \\
\cline{4-4}
& & & > 32767 & \\
\cline{4-4}
& & & < -32768 & \\
\hline
\end{tabular}
\caption{Int converter - Criteria combinations}
\label{tab:int_converter_second}
\end{table}

\paragraph{Boundary conditions}
\begin{itemize}
\item Sign:
\begin{itemize}
\item T1boundary(``0''; 0)
\item T2boundary(``-0''; 0)
\end{itemize}
\item Well formed:
\begin{itemize}
\item T3boundary(``''; error)
\item T4boundary(``-''; error)
\end{itemize}
\item Range:
\begin{itemize}
\item T5boundary(``32767''; 32767)
\item T6boundary(``32768''; error)
\item T7boundary(``32766''; 32767)
\end{itemize}
\end{itemize}

\section{Calendar}
Consider a method that returns the number of days in a month, given the month and year.
\begin{verbatim}
public class MyGregorianCalendar {
  public static int getNumDaysInMonth(int month, int year) {
    ...
  }
  ...
}
\end{verbatim}
The month and the year are specified as integer. By convention, 1 represents January, 2 represents February and so on. The range of valid input for the year is 0 to \texttt{maxInt}. To determine leap years:
\begin{verbatim}
if year modulo 400 is 0 then leap
  else if year modulo 100 is 0 then no_leap
  else if year modulo 4 is 0 then leap
  else no_leap
\end{verbatim}

\paragraph{Equivalence classes}

\begin{itemize}
\item Month:
\begin{itemize}
\item Minint to 0;
\item 30 days \{4, 6, 9, 11\};
\item 31 days \{1, 3, 5, 7, 8, 10, 12\};
\item February \{2\};
\item 13 to Maxint.
\end{itemize}
\item Year:
\begin{itemize}
\item Minint to 0;
\item 1 to Maxint.
\end{itemize}
\end{itemize}
Test cases are shown in table~\ref{tab:calendar_test_cases}. Actually, since the number of month is limited, it is possible to explore all possible combination for each single month, not only for February. In this way, more test cases will be written and a better coverage is guaranteed, but this approach requires more time and effort in writing and running all test cases.

\paragraph{Boundary conditions}
In each partition, boundaries must be considered and test cases on boundary conditions have to be written.

\begin{table}
\centering
\small
\begin{tabular}{|l|l|c|l|}
\hline
\textbf{Month} & \textbf{Year} & \textbf{Leap condition} & \textbf{Test case} \\
\hline
\multirow{2}{*}{Minint to 0} & Minint to 0 & \multirow{2}{*}{-} & T1(-10, -10; error) \\
\cline{2-2} \cline{4-4}
& 1 to Maxint & & T2(-10, 10; error) \\
\hline
\multirow{2}{*}{\{4, 6, 9, 11\}} & Minint to 0 & \multirow{2}{*}{-} & T3(4, -10; error) \\
\cline{2-2} \cline{4-4}
& 1 to Maxint & & T4(6, 2000; 30) \\
\hline
\multirow{2}{*}{\{1, 3, 5, 7, 8, 10, 12\}} & Minint to 0 & \multirow{2}{*}{-} & T5(7, -10; error) \\
\cline{2-2} \cline{4-4}
& 1 to Maxint & & T6(7, 2000; 31) \\
\hline
\multirow{5}{*}{\{2\}} & Minint to 0 & - & T7(2, -10; error) \\
\cline{2-4}
& \multirow{4}{*}{1 to Maxint} & \multicolumn{1}{l|}{Year modulo 400} & T8(2, 2000; 29) \\
\cline{3-4}
& & \multicolumn{1}{l|}{Year modulo 100} & T9(2, 1900; 28) \\
\cline{3-4}
& & \multicolumn{1}{l|}{Year modulo 4} & T10(2, 1904; 29) \\
\cline{3-4}
& & \multicolumn{1}{l|}{Else} & T11(2, 1903; 28) \\
\hline
\multirow{2}{*}{13 to Maxint} & Minint to 0 & \multirow{2}{*}{-} & T12(14, -10; error) \\
\cline{2-2} \cline{4-4}
& 1 to Maxint & & T13(15, 2000; error) \\
\hline
\end{tabular}
\caption{Calendar - Test cases}
\label{tab:calendar_test_cases}
\end{table}

\section{Events}
A queue of events in a simulation system receives events. Each event ha s time tag. It is possible to extract events from the queue, the extraction must return the event with lower time tag. The queue discards events with negative or null time tag and accepts at most 100000 events. Events with the same tag must me merged, i.e.,\@ the second received is disregarded.
\begin{verbatim}
public class EventQueue {
  public void reset();
  public void push(int timeTag) throws InvalidTag, QueueOverflow;
  public int pop() throws EmptyQueue;
}
\end{verbatim}

\subsection{Push}
\begin{itemize}
\item Sign of the time tag ($\le 0$, $> 0$);
\item Number of events (0 to 100000, > 100000);
\item Repetitions of time tags (Yes, No).
\end{itemize}
Test cases are shown in table~\ref{tab:events_push}. Boundary for the time tag has to be considered, too.

\subsection{Pop}
These tests are quite similar to the push ones. Furthermore, they use the \texttt{push} function which has to be already tested, indeed. 

\begin{table}
\centering
\begin{tabular}{|l|l|l|l|}
\hline
\textbf{Time tag} & \textbf{\# Events} & Repetitions & \textbf{Test case} \\
\hline
\multirow{6}{*}{$\le 0$} & \multirow{4}{*}{0 to 100000} & \multirow{3}{*}{No} & T1 \\
\cline{4-4}
& & & TB1 \\
\cline{4-4}
& & & TB2 \\
\cline{3-4}
& & Yes & T2 \\
\cline{2-4}
& \multirow{2}{*}{> 100000} & No & T3 \\
\cline{3-4}
& & Yes & T4 \\
\hline
\multirow{6}{*}{$> 0$} & \multirow{2}{*}{0 to 100000} & No & T5 \\
\cline{3-4}
& & Yes & T6 \\
\cline{2-4}
& \multirow{4}{*}{> 100000} & \multirow{3}{*}{No} & T7 \\
& & & TB3 \\
& & & TB4 \\
\cline{3-4}
& & Yes & T8 \\
\hline
\end{tabular}
\caption{Events - Push test cases}
\label{tab:events_push}
\end{table}

\subsection{Test cases}
\begin{multicols}{2}
\paragraph{T1}
\begin{verbatim}
try {
  EventsQueue eq =
    new EventsQueue();
  eq.push(-10);
  eq.push(-12);
  eq.push(-14);
} catch() {
  InvalidTag
  success();
}
\end{verbatim}

\paragraph{T2}
\begin{verbatim}
try {
  EventsQueue eq =
    new EventsQueue();
  eq.push(-10);
  fail();
  eq.push(-10);
  eq.push(-10);
} catch() {
  InvalidTag
  success();
}
\end{verbatim}

\paragraph{T3}
\begin{verbatim}
try {
  EventsQueue eq =
    new EventsQueue();
  int i = 1;
  for(100010 times) {  
    eq.push(i--);
  }
  fail();
} catch() {
  InvalidTag or QueueOverflow
  success();
}
\end{verbatim}

\bigskip
\paragraph{T4}
\begin{verbatim}
try {
  EventsQueue eq =
    new EventsQueue();
  int i = 1;
  for(100010 times) {  
    eq.push(-10);
  }
  fail();
} catch() {
  InvalidTag or QueueOverflow
  success();
}
\end{verbatim}

\paragraph{T5}
\begin{verbatim}
try {
  EventsQueue eq =
    new EventsQueue();
  eq.push(10);
  eq.push(20);
  eq.push(30);
} catch() {
  fail();
}
eq.pop() == 10;
eq.pop() == 20;
eq.pop() == 30;
eq.pop() -> EmptyQueue;
\end{verbatim}

\paragraph{T6}
\begin{verbatim}
try {
  EventsQueue eq =
    new EventsQueue();
  eq.push(10);
  eq.push(10);
  eq.push(10);
} catch() {
  fail();
}
eq.pop() == 10;
eq.pop() -> EmptyQueue;
\end{verbatim}

\paragraph{T7}
\begin{verbatim}
try {
  EventsQueue eq =
    new EventsQueue();
  int i = 1;
  for(100010 times) {  
    eq.push(i++);
  }
  fail();
} catch() {
  QueueOverflow
  success();
}
\end{verbatim}

\paragraph{T8}
\begin{verbatim}
try {
  EventsQueue eq =
    new EventsQueue();
  for(100010 times) {  
    eq.push(10);
  }
} catch() {
  fail();
}
\end{verbatim}

\paragraph{TB1 - Empty and tag = 0}
\begin{verbatim}
try {
  EventsQueue eq =
    new EventsQueue();
  eq.push(0);
} catch() {
  InvalidTag
  success();
}
\end{verbatim}

\paragraph{TB2 - Tag = 0}
\begin{verbatim}
try {
  EventsQueue eq =
    new EventsQueue();
  eq.push(10);
  eq.push(0);
} catch() {
  InvalidTag
  success();
}
\end{verbatim}

\paragraph{TB3}
\begin{verbatim}
try {
  EventsQueue eq =
    new EventsQueue();
  int i = 1;
  for(100000 times) {  
    eq.push(i++);
  }
  pop() == 1;
} catch() {
  fail();
}
\end{verbatim}

\paragraph{TB4}
\begin{verbatim}
try {
  EventsQueue eq =
    new EventsQueue();
  int i = 1;
  for(100001 times) {  
    eq.push(i++);
  }
  fail();
  pop() == 1;
} catch() {
  success();
}
\end{verbatim}
\end{multicols}